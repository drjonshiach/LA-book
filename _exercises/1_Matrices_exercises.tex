\documentclass[a4paper,11pt]{article}

% Packages
\usepackage{graphicx}
\usepackage{amsmath}
\usepackage{amssymb}
\usepackage[margin=2cm]{geometry}
\usepackage{enumitem}
\usepackage{tasks}
\usepackage{svg}

% Title info
\author{Dr Jon Shiach}
\date{Semester 1}

% Tasks package
\usepackage{tasks}
\DeclareInstance{tasks}{alphabetize-parens}{default}{
    label = (\alph*),
    label-width    = 2em,
}
\settasks{style=alphabetize-parens}

% Commands
\renewcommand{\vec}{\mathbf}

\title{Matrices Exercises}

\begin{document}

\maketitle

\begin{enumerate}[label=1.\arabic*]
    \item
    \begin{enumerate}
        \item Write down the $3 \times 3$ matrix $A$ whose entries are given by $a_{ij} = i+j.$
        \item Write down the $4 \times 4$ matrix $B$ whose entries are given by $b_{ij} = (-1)^{i+j}.$
        \item Write down the $4 \times 4$ matrix $C$ whose entries are given by 
        $$c_{ij} = 
            \begin{cases}
            -1, & i>j, \\
            0, & i=j, \\
            1, & i<j. \\
            \end{cases} $$
    \end{enumerate}

    \item The Hilbert matrix is the $n \times n$ matrix $H$ where the value of its elements are $h_{ij} = \dfrac{1}{i + j - 1}$. 
    \begin{enumerate}
        \item Write down the $4 \times 4$ Hilbert matrix. 
        \item Show that an $n \times n$ Hilbert matrix is symmetric. 
    \end{enumerate}

    \item Given the matrices
    \begin{align*}
        A &= \begin{pmatrix} 1 & -3 \\ 4 & 2 \end{pmatrix}, &
        B &= \begin{pmatrix} 3 & 0 \\ -1 & 5 \end{pmatrix}, \\
        C &= \begin{pmatrix} 5 \\ 9 \end{pmatrix}, &
        D &= \begin{pmatrix} 1 & 1 & 3 \\ 4 & -2 & 3 \end{pmatrix}, \\
        E &= \begin{pmatrix} 1 & 2 \\ 0 & 6 \\ -2 & 3 \end{pmatrix} &
        F &= \begin{pmatrix} 1 & -2 & 4 \end{pmatrix}, \\
        G &= \begin{pmatrix} 4 & 2 & 3 \\ -2 & 6 & 0 \\ 0 & 7 & 1 \end{pmatrix}, &
        H &= \begin{pmatrix} 1 & 0 & 1 \\ 5 & 2 & -2 \\ 2 & -3 & 4 \end{pmatrix}.
    \end{align*}
    Calculate the following where possible:
    \begin{tasks}(4)
        \task $A + B$
        \task $B + C$
        \task $A^\mathsf{T}$
        \task $C^\mathsf{T}$
        \task $3B - A$
        \task $(F^\mathsf{T})^\mathsf{T}$
        \task $A^\mathsf{T} + B^\mathsf{T}$
        \task $(A + B)^\mathsf{T}$
    \end{tasks}

    \item Using the matrices from exercise 1.3 calculate the following where possible:
    \begin{tasks}(4)
        \task $AB$
        \task $BA$
        \task $AC$
        \task $CA$
        \task $C^\mathsf{T}C$
        \task $CC^\mathsf{T}$
        \task $DE$
        \task $GH$
        \task $A(DE)$
        \task $(AD)E$
        \task $A^3$
        \task $G^4$
    \end{tasks}

    \item Calculate the determinants of the square matrices from exercise 1.3.
    
    \item For each non-singular matrix from exercise 1.3 calculate its inverse. Show that your answers are correct.
    \item Show that $AA^\mathsf{T}$ is a symmetric matrix. Hint: use the properties of matrix transpose.
    
    \item Show that $(AB)^{-1} = B^{-1}A^{-1}$. Hint: use the associativity law. 
    
    \item If $A$ and $B$ are $n \times n$ matrices is the following equation true?
    $$(A + B)^2 = A^2 + 2AB + B^2$$
    If not, under what conditions would it be true?

    \item An involutory matrix is a matrix that is its own inverse, i.e., it satisfies the equation $A^2 = I$. Under what conditions is the following matrix an involutory matrix?
    $$A = \begin{pmatrix} a & b \\ c & -a \end{pmatrix} $$

    \item Which of the following statements are true? For the false statements, give one counter example where the statement doesn't hold.
    \begin{enumerate}
        \item If $A = B$ then $AC = BC$
        \item If $AC = BC$ then $A = B$
        \item For $[O]_{ij} = 0$, if $AB = O$ then $A = O$ or $B = O$
        \item If $A + C = B + C$ then $A = B$
        \item If $A^2 = I$ then $A = \pm I$
        \item If $B = A^2$ and if $A$ is an $n \times n$ symmetric matrix then $b_{ii} \geq 0$ for $i = 1, 2, \ldots, n$
        \item If $AB = C$ and if two of the matrices are square then so is the third
        \item If $AB = C$ and if $C$ has a single column then so does $B$
        \item If $A^2 = I$ then $A^n = I$ for all integers $n \geq 2$
    \end{enumerate}

    \item Given the matrices
    \begin{align*}
        A &= \begin{pmatrix} 1 & -3 \\ 4 & 2 \end{pmatrix}, &
        B &= \begin{pmatrix} 3 & 0 \\ -1 & 5 \end{pmatrix},
    \end{align*}
    solve the following equations for $X$.
    \begin{tasks}(4)
        \task $5X = A$
        \task $X + A = I$
        \task $2X - B = A$
        \task $XA = I$
        \task $BX = A$
        \task $A^2 = X$
        \task $X^2 = B$
        \task $(X + A)B = I$
    \end{tasks}

\end{enumerate}

\end{document}